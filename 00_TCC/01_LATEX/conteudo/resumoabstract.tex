%!TeX root=../tese.tex
%("dica" para o editor de texto: este arquivo é parte de um documento maior)
% para saber mais: https://tex.stackexchange.com/q/78101

% As palavras-chave são obrigatórias, em português e em inglês, e devem ser
% definidas antes do resumo/abstract. Acrescente quantas forem necessárias.
\palavraschave{Palavra-chave1, Palavra-chave2, Palavra-chave3}

\keywords{Keyword1,Keyword2,Keyword3}

% O resumo é obrigatório, em português e inglês. Estes comandos também
% geram automaticamente a referência para o próprio documento, conforme
% as normas sugeridas da USP.
\resumo{
Este trabalho apresenta um estudo do Modelo Lorenz 80, originalmente proposto em \citet{Lorenz1980}, a partir de uma abordagem estocástica inspirada em \citet{Chekroun2021}. São abordados fundamentos teóricos essenciais, como o formalismo de Mori-Zwanzig, propriedades gerais de equações diferenciais estocásticas e as características matemáticas e físicas do próprio modelo. O desenvolvimento computacional inclui a implementação e simulação numérica do modelo utilizando as linguagens \textit{Julia} e \textit{Python}, com ênfase em bibliotecas científicas voltadas para sistemas dinâmicos estocásticos e análise de dados. Por fim, realiza-se uma análise exploratória de diferentes configurações do termo de ruído, com o objetivo de investigar abordagens alternativas e complementares ao tratamento original.
}

\abstract{
This project presents a study of the Lorenz 80 Model, originally proposed in \citet{Lorenz1980}, from a stochastic approach inspired by \citet{Chekroun2021}. Essential theoretical foundations are covered, such as the Mori-Zwanzig formalism, general properties of stochastic differential equations and the mathematical and physical characteristics of the model itself. The computational development includes the implementation and numerical simulation of the model using the \textit{Julia} and \textit{Python} languages, with an emphasis on scientific libraries aimed at stochastic dynamic systems and data analysis. Finally, an exploratory analysis of different configurations of the noise term is carried out, with the aim of investigating alternative and complementary approaches to the original treatment.
}
