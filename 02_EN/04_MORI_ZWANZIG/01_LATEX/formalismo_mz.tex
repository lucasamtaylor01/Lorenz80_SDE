\documentclass[12pt]{article}
% Encoding and Language
\usepackage[utf8]{inputenc}
\usepackage[portuguese]{babel}
% Page Formatting and Text Style
\usepackage[a4paper, margin=2cm]{geometry}
\usepackage{parskip}
\linespread{1.5}
\usepackage{csquotes}
% Links
\usepackage{xcolor}
\usepackage[hidelinks]{hyperref}
% Mathematics
\usepackage{amsmath, amssymb, amsfonts}
\usepackage{cancel}
\usepackage{tikz}
% Bibliography (ABNT)
\usepackage[backend=biber, style=abnt, language=portuguese, natbib=true]{biblatex}
\addbibresource{ref.bib}
% Other Packages
\usepackage{graphicx}
\usepackage{ascii}
\usepackage{listings}
\usepackage{enumitem}
\usepackage{url}

\title{Introduction to the Mori-Zwanzig Formalism}
\date{April 2025}
\author{
    \textbf{Student:} Lucas Amaral Taylor\\
    \textbf{Advisor:} Prof. Dr. Breno Raphaldini Ferreira da Silva
}
\begin{document}
\maketitle
\section{Introduction}
The main objective of the article by \citet{Chekroun2021} is to simplify the Lorenz 80 model while preserving its behavior. To do this, we will use the Mori-Zwanzig method, which is a physical-statistical approach applicable in systems such as L80.
The Mori-Zwanzig method, developed by Robert Walter Zwanzig and Hajime Mori in the second half of the 20th century, is used in Hamiltonian systems. This method consists of classifying the system's variables into two categories: “resolved” and “unresolved.” Resolved variables are those whose behavior and values are well known, while unresolved variables are those for which no direct information is available. To replace these unresolved variables, the method introduces stochastic terms, called noise, as well as a damping term, also known as a memory term. This approach allows the behavior of the system of interest to be adequately preserved, even without fully knowing the unresolved variables.
Given the relevance of this method to the work of \citet{Chekroun2021}, we chose to include an introduction to the Mori-Zwanzig formalism in order to provide a better understanding of its application in the context of the Lorenz 80 model and to allow for a solid theoretical basis for further exploration.
\newpage
\section{Motivation}
Consider the example from \citet[p.173]{Chorin2013}: a system with two particles in one spatial dimension, with Hamiltonian given by:
\begin{equation*}
    H = \frac{1}{2}(q_1^2 + q_2^2 + q_1^2 q_2^2 + p_1^2 + p_2^2)
\end{equation*}
where $q_i$ and $p_i$, $i = 1, 2$, represent positions and momenta. Once in motion, harmonic oscillators oscillate indefinitely. The equations of motion are expressed by:
\begin
{align}
    \dot{q}_1 & = p_1, \nonumber             \\
    \dot{p}_1 & = -q_1(1 + q_2^2), \nonumber \\
    \dot{q}_2 & = p_2, \nonumber             \\
    \dot{p}_2 & = -q_2(1 + q_1^2).
               
\label{eq:example-harmonic-system}
\end{align}
Suppose that the initial values $q_1(0)$ and $p_1(0)$ are known. Assume that $q_2(0)$ and $p_2(0)$ are sampled from the probability density function:
\begin{equation*}
    W = \frac{e^{-H(q,p)}}{Z}
\end{equation*}
This sampling can be performed, for example, using the Monte Carlo method via Markov chains.
Given a sample of $q_2(0)$ and $p_2(0)$, the system \eqref{eq:example-harmonic-system} can be solved. However, for each new sample of $q_2(0)$ and $p_2(0)$, a distinct trajectory is obtained for $q_1(t)$ and $p_1(t)$. In particular, one may want to calculate the expected values of $q_1$ and $p_1$ at time $t$, given their initial values, which represent the best estimates of $q_1(t)$ and $p_1(t)$:
\begin{equation*}
    
\mathbb{E}[q_1(t)\mid q_1(0), p_1(0)], \quad \mathbb{E}[p_1(t)\mid q_1(0), p_1(0)].
\end{equation*}
Once $q_2$ and $p_2$ have been sampled, the complete system of four equations can be solved. This can be done repeatedly, allowing the average values of $q_1(t)$ and $p_1(t)$ to be calculated over several runs.
However, there is a limitation: this approach is only suitable for the initial instants of the system. As time progresses, the expected value of $q_1(t)$ deviates from the actual value, compromising the accuracy of the approximation.
\newpage
\section{Preliminaries}
\subsection{Writing nonlinear ODE systems as systems of linear PDEs}
Consider the system of ordinary differential equations (ODEs) given by:
\begin{equation}
    \frac{d}{dt} \phi(x,t) = R(\phi(x,t)), \quad \phi(x, 0) = x,
    \label{eq:sistema-edo}
\end{equation}
where $R$ is a nonlinear function, $\phi$ is a time-dependent function, and $R$, $\phi$, and $x$ can take on infinite dimensions, being formed by the vectors $R_i$, $\phi_i$, and $x_i$, respectively.
From this, we can define the \textit{Liouville operator} associated with equation \eqref{eq:sistema-edo} as:
\begin{equation}
    L = \sum_i R_i(x) \frac{\partial}{\partial x_i}
    \label{eq:liouville-operator}
\end{equation}
Using the \textit{Liouville Operator}, we can transform the system of nonlinear ODEs into a system of linear partial differential equations (PDEs) of the form:
\begin{equation}
    
u_t = Lu, \quad u(x,0) = g(x)
    \label{eq:linear-pde-system}
\end{equation}
The solution to this system exists, is unique, and is given by:
\begin{equation}
    u(x,t) = g(\phi(x,t))   
    \label{eq:solution-linear-pde-system}
\end{equation}
Therefore, we have that equation \eqref{eq:linear-pde-system} is well defined\footnote{Details of the proof can be found in \citet[p.~181-182]{Chorin2013}}.
\subsection{Semigroup notation}
\subsection{Definition of semigroup}
Let $X$ be a non-empty set, equipped with a binary operation $ * $, that is, $ X \times X \to X $, which satisfies the associativity property:
\begin{equation*}
    
(a * b) * c = a * (b * c), \quad \forall a,b,c \in X.
\end{equation*}
\subsection{Introduction to notation}
Semigroup notation offers a compact and efficient way to represent solutions to differential equations, particularly partial or evolution equations.
Consider the operator $\Delta$ defined by:
\begin{equation*}
    \Delta \psi = \psi_{xx}, \quad \text{where $\psi$ is a smooth function}.
\end{equation*}
Now, consider the differential equation:
\begin{equation*}
    
\frac{dv}{dt} - kv = 0, \quad v(0) = v_0,
\end{equation*}
whose solution is well known: $ v(t) = v_0 e^{kt} $.
Similarly, consider the heat equation:
\begin{equation*}
    v_t - \frac{1}{2} \Delta v = 0, \quad v(x, 0) = \phi(x),
\end{equation*}
where $v_t$ is the derivative of $v$ with respect to time and $\phi(x)$ is the initial condition. Instead of solving directly, we express the solution using semigroup notation:
\begin{equation*}
    
v(t) = e^{\frac{1}{2}t \Delta} \phi.
\end{equation*}
Here, $\mathbb{E}^{\frac{1}{2}t \Delta}$ is a semigroup operator generated by the diffusion operation (by the operator $\Delta$). It is applied to the initial condition $\phi(x)$, and the solution $v(t)$ describes the temporal evolution of $v(x,t)$ over time $t$. This notation allows us to represent solutions to differential equations in a compact way, exploiting the associative structure of the semigroup operation. Specifically, it satisfies the composition property:
\begin{equation*}
    e^{\frac{1}{2}(t+s)\Delta} = e^{\frac{1}{2}t \Delta} e^{\frac{1}{2}s \Delta}.
\end{equation*}
\subsection{Application of notation}
Given the semigroup notation presented above, we apply this notation to equation \eqref{eq:solution-linear-pde-system}:
\begin{equation}
e^{tL}g(x) = g(\phi(x,t))
\label{eq:solution-linear-pde-system-semigroup}
\end{equation}
Note that $\mathbb{E}^{tL}x$ does not represent a direct evaluation of $\mathbb{E}^{tL}$, but rather the action of the operator $\mathbb{E}^{tL}$ on the vector formed by the components $x_i$. Furthermore, the function $g$ commutes with the temporal variation of the initial conditions of $x_i$.
 
It is worth noting that $g$ is a time-independent function with respect to the variables that describe the physical system, and its variation occurs exclusively due to the change of these variables over time. Thus, equation \eqref{eq:linear-edp-system} can be expressed as:
\begin{equation}
    Le^{tL} = e^{tL}L
\end{equation}
This same relationship applies to matrices: let $A$ and $B$ be two matrices, then the following identity is valid:
\begin{equation}
 
\exp(t(A+B)) = \exp(tA) + \int_0^t \exp\left((t-s)(A+B)\right)B\exp(sA) \,
{equation}
    \exp(t(A+B)) = \exp(tA) + \int_0^t \exp\left((t-s)(A+B)\right)B\exp(sA) \, ds
\label{eq:formula-de-duhamel}
\end{equation}
This equation, known as the \textit{Duhamel formula} or \textit{Dyson formula}, is well defined.
\subsection{Hermitian Polynomials}
First, we define the inner product as:
\begin{equation}
    \langle u, v \rangle = \int_{-\infty}^{+\infty} \frac{e^{-x^2/2}}{\sqrt{2 \pi}} u(x) v(x) \, dx
    \label{eq:inner-product-hermitian-one-dimensional}
\end{equation}
The polynomials $p_n(x)$ and $p_m(x)$ are orthonormal with respect to this inner product \eqref{eq:inner-product-hermitian-one-dimensional} when they satisfy the following condition:
\begin{equation*}
    
\langle p_n, p_m \rangle = \int_{-\infty}^{\infty} e^{-x^2/2} p_n(x) p_m(x) \, dx = \delta_{nm},
\end{equation*}
where $\delta_{nm}$ is the Kronecker delta, which has the following properties:
\begin{enumerate}
    
\item \textbf{Orthogonality}: For $n \neq m$, the polynomials are orthogonal, that is, their inner product is zero:
          \begin{equation*}
\langle p_n, p_m \rangle = 0 \quad \text{when} \quad n \neq m
\end{equation*}
                          
\item \textbf{Normalization}: For $n = m$, the polynomials are normalized, so that the inner product is equal to 1:
          \begin{equation*}
            
\langle p_n, p_n \rangle = 1
          \end{equation*}
\end{enumerate}
In the $n$-dimensional case, the inner product is generalized to:
\begin{equation*}
\langle u, v \rangle = \int_{-\infty}^{+\infty} \cdots \int_{-\infty}^
{+\infty} (2 \pi)^{-n/2} \exp \left(-\sum_{i=1}^n \frac{x_i^2}{2} \right) u(x) v(x) \, dx_1 \ldots dx_n
\end{equation*}
More generally, if $H(q, p)$ is a Hamiltonian, it is possible to define a family of polynomials in the variables $q$ and $p$ that are orthonormal with respect to the canonical density $Z^{-1} e^{-H/T}$. Polynomials that satisfy this condition are also called \textit{Hermitian polynomials}.
Finally, for the Mori-Zwanzig formalism, we will consider an $n$-dimensional space $\Gamma$ with a given probability density. We will divide the coordinates into two types: $\hat{x}$ and $\tilde{x}$. Let $g$ be a function of $x$; then $\mathbb{P}g = \mathbb{E}[g \mid \hat{x}]$ is an orthogonal projection onto the subspace of functions of $\hat{x}$. We have that this projection generates a subspace of Hermitian polynomials that are functions of $\hat{x}$ and projecting onto these polynomials.
\newpage
\section{Mori-Zwanzig}
\subsection{Construction}
Let us take again the system \eqref{eq:sistema-edo}, reproduced below:
\begin{equation*}
    \frac{d}{dt} \phi(x,t) = R(\phi(x,t)), \quad \phi(x, 0) = x,
\end{equation*}
Recall that the equation is composed of $n$-dimensional components. Among these $n$ components, we define the first $m$ components of $\phi$, with $m < n$, as the variables of interest. We then classify $\hat{\phi}$ as the “solved” variables and $\tilde{\phi}$ as the “unsolved” variables:
\begin{equation*}
    \phi = (\hat{\phi}, \tilde{\phi}), \quad \hat{\phi} = \left(\phi_1, \ldots, \phi_m \right), \quad \tilde{\phi} = \left(\phi_{m+1}, \ldots, \phi_n\right)
\end{equation*}
The same applies to $x$ and $R$: $x = (\hat{x}, \tilde{x})$ and $R = (\hat{R}, \tilde{R})$. From the solved variables, we seek to create predictions for the model of interest, using the solutions from one part of the equation.
Based on the \textit{Liouville operator} and \textit{semigroup notation}, we can rewrite the components of $\hat{\phi}$ as\footnote{Note that each component depends on \textbf{all} values of $x$. Therefore, if $\tilde{x}$ is random, then $\hat{\phi}$ will also be random.}:
\begin{equation*}
\hat{\phi}_j(x,t) = e^{tL}x_j, \quad 1 \leq j \leq m
\end{equation*}
Still in semigroup notation, the equation of these components is given by:
\begin{equation}
    \frac{\partial}{\partial t}e^{tL}x_j = Le^{tL}x_j = e^{tL}Lx
    \label{eq:mori-zwanzig-semigroup-notation}
\end{equation}
From the \textit{orthogonal projection} introduced in the previous section, we define $\mathbb{P}$ as the projection given by: $\mathbb{P}g(x) = \mathbb{E}[g|\hat{x}]$. We assume that, at time $t = 0$, we know the joint density of all variables $x$, but only the initial data $\hat{x}$ are known. The density of the variables in $\tilde{x}$ is then the joint density of all variables $x$ with $\hat{x}$ fixed. Thus, $\mathbb{P}$ is a projection onto a space of functions with fixed variables and, therefore, independent of time.
The projections $\mathbb{P}\hat{\phi}(t) = \mathbb{E}[\hat{\phi}(t) | \hat{x}]$ are of greatest interest to us, as they estimate the behavior of the system from a reduced set of variables.
Defining $\mathbb{Q} = I - \mathbb{P}$ and considering that the following properties are valid for any orthogonal projections:
\begin{enumerate}
    \item $\mathbb{P}^2 = \mathbb{P}$;
    
\item $\mathbb{Q}^2 = \mathbb{Q}$;
    \item $\mathbb{P}\mathbb{Q} = 0$.
\end{enumerate}
We can rewrite equation \eqref{eq:mori-zwanzig-semigroup-notation} as:
\begin{equation}
\frac{\partial }{\partial t}e^{tL}x_j = e^{tL}\mathbb{P}Lx_j + e^{tL}\mathbb{Q}Lx_j
\label{eq:mori-zwanzig-pre-mz}
\end{equation}
Now using the \textit{Dyson formula}, with $A = \mathbb{Q}L$ and $B = \mathbb{Q}L$, we obtain:
\begin{equation}
    e^{tL} = e^{t\mathbb{Q}L} + \int_0^t e^{(t-s)L} \mathbb{P}L e^{s\mathbb{Q}L} \, ds
    
\label{eq:mori-zwanzig-dyson-mz}
\end{equation}
Due to the linearity of the Liouville equation and from equations \eqref{eq:mori-zwanzig-pre-mz} and \eqref{eq:mori-zwanzig-dyson-mz}, we obtain:
\begin{equation}
    \frac{\partial}{\partial t} e^{tL} x_j = e^{tL} \mathbb{P}L x_j + e^{t\mathbb{Q}L} \mathbb{Q}L x_j + \int_0^t e^{(t-s)L} \mathbb{P}L e^{s\mathbb{Q}L} \mathbb{Q}L x_j \, ds
    
\label{eq:mori-zwanzig}
\end{equation}
The above equation expresses the \textit{Mori-Zwanzig equation}.
\subsection{Term-by-term analysis}
\subsubsection{First term}
The first term is given by:
\begin{equation}
    
e^{tL} \mathbb{P}L x_j
    \label{eq:first-term-mz}
\end{equation}
Note that:  
\begin{equation*}
    Lx_j = \sum_i R_i\left(\frac{\partial}{\partial x_i}\right)x_j = R_j(x)
\end{equation*}
Therefore,  
\begin{equation*}
    \mathbb{P}Lx_j = \mathbb{E}[R_j(x)\,|\,\hat{x}] \quad \text{Note that this is a function exclusively of $\hat{x}$.}
\end{equation*}
With this, we can conclude that:  
\begin{equation*}
    
e^{tL}\mathbb{P}Lx_j = \bar{R}_j\left(\hat{\phi}(x,t)\right)
\end{equation*}
Furthermore, the first term represents the system's own dynamics in the solved variables. In addition, it is a Markovian term, as it depends only on the current state of the system at time $t$.
\subsubsection{Second term}
For the second term, we define:
\begin{equation*}
w_j = e^{t\mathbb{Q}L} \mathbb{Q}L x_j
\end{equation*}
By definition, we have:
\begin{align}
\frac{\partial}{\partial t} w_j(x,t) & = \mathbb{Q}L w_j(x,t),\\
    
w_j(x,0)                             & = \mathbb{Q}L x_j = (I - \mathbb{P})R_j(x) = R_j(x) - \mathbb{E}[R_j \mid \hat{x}]. 
    \label{eq:mori-zwanzig-orthogonal-dynamics}
\end{align}
Note that $w_j(x,0) = \mathbb{Q}Lx_j = R_j(x) - \mathbb{E}[R_j(x) \mid \hat{x}]$ represents the \textit{floating part} of the variable $R_j(x)$, that is, the unpredictable component given $\hat{x}$. This part evolves according to the \textit{orthogonal dynamics}, so that $\mathbb{P} w_j(x,t) = 0$ for all $t$, keeping the term as purely unresolved noise over time.
More specifically, the noise subspace is formed by the components of the functions that are orthogonal to the functions of $\hat{x}$, generally corresponding to terms that depend on $\tilde{x}$.
 
\subsubsection{Third term}
The third term, given by:
\begin{equation*}
    \int_0^t e^{(t-s)L} \mathbb{P}L e^{s\mathbb{Q}L} \mathbb{Q}L x_j
\end{equation*}
is classified as the memory term, since it involves the integration of quantities that depend on states prior to the current one.
Let $\mathbb{P}$ be projected onto the span of the Hermitian polynomials $H-1, H_2, \ldots$ with arguments in $\hat{x}$. Thus, for a given function $\psi$, we have that: $\mathbb{P}\psi = \sum (\psi, H_k)H_k$, thus, we have:
\begin{align*}
    \mathbb{P}Le^{s\mathbb{Q}L} \mathbb{Q}Lx_j & = \mathbb{P}L(\mathbb{P} + \mathbb{Q})e^{s\mathbb{Q}L} \mathbb{Q}Lx_j                           \\
                                               & = \mathbb{P}L\mathbb{Q}e^{s\mathbb{Q}L} \mathbb{Q}Lx_j                                          \\
                                               & = \sum_k \langle L\mathbb{Q}e^{s\mathbb{Q}L} \mathbb{Q}Lx_j, H_k(\hat{x}) \rangle H_k(\hat{x}). 
\end{align*}
The inner product is defined as an expected value with respect to the initial probability density. Let us assume that $L$ is antisymmetric, that is, $(u, Lv) = -(Lu, v)$, then:
\begin{align*}
    (L Q e^{s Q L} Q L x_j, H_k(\hat{x})) 
      & = - (Q e^{s Q L} Q L x_j, L H_k)  \\
      & = - (e^{s Q L} Q L x_j, Q L H_k).
\end{align*}
Both $Q L x_j$ and $Q L H_k$ are in the noise subspace, and $\mathbb{E}^{s Q L} Q L x_j$ is a solution at time $s$ of the orthogonal dynamics equation with data in the noise subspace; $P L e^{s Q L} Q L x_j$ is then a \textbf{sum of temporal noise covariances}.
\newpage
\section{Returning to the motivation}
\subsection{Introduction}
First, let's recall the problem at hand: we have a Hamiltonian system, with the Hamiltonian defined by  
\begin{equation*}
    H = \frac{1}{2}(q_1^2 + q_2^2 + q_1^2 q_2^2 + p_1^2 + p_2^2).
\end{equation*}
The associated equations of motion are given by:
\begin{align*}
\dot{q}_1 & = p_1, \nonumber             \\
\dot{p}_1 & = -q_1(1 + q_2^2), \nonumber \\
    
\dot{q}_2 & = p_2, \nonumber             \\
    \dot{p}_2 & = -q_2(1 + q_1^2).
\end{align*}
Applying the \textit{Liouville operator}, we obtain:
\begin{equation*}
    
L = p_1 \frac{\partial}{\partial q_1} - q_1(1 + q_2^2)\frac{\partial}{\partial p_1}
    + p_2 \frac{\partial}{\partial q_2} - q_2(1 + q_1^2)\frac{\partial}{\partial p_2}.
\end{equation*}
As before, we assume that the initial values $q_1(0)$ and $p_1(0)$ are known, while $q_2$ and $p_2$ are defined by a probability density function given by:
\begin{equation*}
    
W(x) = \exp\left(\frac{-H(q_1,p_1,q_2,p_2)}{Z}\right) \quad \text{(a canonical density with temperature $T = 1$).}
\end{equation*}
Our goal is to evaluate $q_1(t)$ and $p_1(t)$ from the Mori-Zwanzig equations \eqref{eq:mori-zwanzig}.
\subsection{First approximation}
Given the nature of the system, it is necessary to make an approximation to introduce the memory term. We know that the orthogonal dynamics described in \eqref{eq:mori-zwanzig-orthogonal-dynamics} are approximately equivalent to the complete dynamics, since they are not sensitive to the solved variables. Based on this, the approximation consists of replacing $e^{t\mathbb{Q}L}$ with $e^{tL}$ in the memory term. This substitution is valid when the influence of the resolved variables on the unresolved ones is small\footnote{This substitution is justified when the coupling between the resolved and unresolved variables is weak.}. 
That said, we can begin the algebraic manipulation. An operator commutes with any function of itself, so that:
\begin{equation*}
\mathbb{Q}Le^{s\mathbb{Q}L} = e^{s\mathbb{Q}L} \mathbb{Q}L.
\end{equation*}
Using this identity and substituting $e^{s\mathbb{Q}L} \to e^{sL}$ on the right side of the equality, we obtain:
\begin{equation*}
    \mathbb{P}Le^{s\mathbb{Q}L} \approx Le^{sL} - e^{sL} \mathbb{Q}L.
\end{equation*}
Then,
\begin{equation*}
    e^{(t-s)L} \mathbb{P}Le^{s\mathbb{Q}L} \approx e^{(t-s)L} Le^{sL} - e^{(t-s)L} e^{sL} \mathbb{Q}L = e^{tL} \mathbb{P}L,
\end{equation*}
which makes the integrand in the integral term of the Mori-Zwanzig equation independent of $s$, resulting in:
\begin{equation*}
    \int_0^t e^{tL} \mathbb{P}L \mathbb{Q}L x_j \, ds = t e^{tL} \mathbb{P}L \mathbb{Q}L x_j,
\end{equation*}
where $\hat{x}$ is the vector with components $x_1 = q_1$ and $x_2 = p_1$. The memory term has thus been reduced to a differential operator multiplied by time $t$. Note that $t = 0$ corresponds to the instant when the initial values of $q_1(t)$ and $p_1(t)$ are assigned, that is, when there is no uncertainty. This approximation can also be justified by assuming that the integrand in the memory term is approximately constant in $s$, and therefore can be evaluated at $s = 0$.
The equations with the simplified integral term constitute the so-called $t$-model. Combining the terms, we have:
\begin{equation*}
    
\frac{d}{dt} e^{tL} \hat{x} = e^{tL} \mathbb{P}L \hat{x} + t e^{tL} \mathbb{P}L \mathbb{Q}L \hat{x} + e^{tL} \mathbb{Q}L \mathbb{Q}L x_j.
\end{equation*}
In the particular case considered, with components $x_j = q_1$ and $x_j = p_1$, we obtain:
\begin{align*}
Lq_1                        & = p_1, \\
\mathbb{P}Lq_1              & = p_1, \\
\mathbb{Q}Lq_1              & = 0,   \\
    
L\mathbb{Q}Lq_1             & = 0,   \\
    \mathbb{P}L \mathbb{Q}L q_1 & = 0,  
\end{align*}
and
\begin{align*}
    
Lp_1 &= -q_1(1 + q_2^2),\\
    \mathbb{P}Lp_1 &= -q_1\left(1 + \frac{1}{1 + q_1^2}\right),\\
    \mathbb{Q}Lp_1 &= -q_1(1 + q_2^2) + q_1\left(1 + \frac{1}{1 + q_1^2}\right),\\
    L\mathbb{Q}Lp_1 &= p_1\left(- (1 + q_2^2) + \left(1 + \frac{1}{1 + q_1^2} \right) \right) - \frac{2q_1^2}{(1 + q_1^2)^2} - 2q_1 q_2 p_2, \\
    \mathbb{P}L\mathbb{Q}Lp_1 & = -\frac{2q_1^2 p_1}{(1 + q_1^2)^2}.
\end{align*}
Thus, the approximate equations of motion for $q_1$ and $p_1$ become:
\begin{align}
    \frac{d}{dt} q_1 & = p_1, \nonumber \\
\frac{d}{dt} p_1 & = -q_1\left(1 + \frac{1}{1 + q_1^2}\right) - 2t\frac
{q_1^2 p_1}{(1 + q_1^2)^2} + e^{tL} \mathbb{Q}L \mathbb{Q}L p_1. \label{eq:mori-zwanzig-hamiltonian-system}
\end{align}
Now suppose we only want to know the quantities:
\begin{equation*}
    \mathbb{E}[q_1(t) \mid q_1(0), p_1(0)], \quad \mathbb{E}[p_1(t) \mid q_1(0), p_1(0)],
\end{equation*}
that is, the conditional expectations of $q_1(t)$ and $p_1(t)$ given $q_1(0), p_1(0)$.
An equation for these quantities can be obtained by applying the operator $\mathbb{P}$ to equations \eqref{eq:mori-zwanzig-hamiltonian-system}, remembering that, by definition, $\mathbb{P}q_1(t) = \mathbb{E}[q_1(t) \mid q_1(0), p_1(0)]$, and similarly for $p_1(t)$. In this process, the noise term disappears.
 However, a problem arises: the mean of a function of one variable is generally not equal to the function of the mean. To overcome this difficulty, an additional simplification is necessary. This complication can be avoided by considering specific trajectories of the solved variables, where such a distinction becomes irrelevant, allowing the approximate equations to be applied directly without the difficulties associated with orthogonal dynamics.
\subsection{Second approximation}
Assume that, for the functions on the right-hand side of equations \eqref{eq:mori-zwanzig-hamiltonian-system}, the averaging operation and the evaluation of the function commute. This means that, for example,
\begin{equation*}
\mathbb{E}[(1 + q_1^2(t))^{-1} 
\mid q_1(0), p_1(0)] \approx \left(1 + \mathbb{E}[q_1(t) \mid \cdot]^2\right)^{-1}.
\end{equation*}
This mean field approximation is valid when the noise is sufficiently small. In the limiting case where the noise is zero, the approximation becomes exact. In the specific problem considered, this assumption should be reasonable if the initial data are sampled from a canonical density with low temperature. Here, we use an initial temperature \(T = 1\).
With that, we define:
\begin{equation*}
    
Q_1(t) = \mathbb{E}[q_1(t) \mid q_1(0), p_1(0)], \quad P_1(t) = \mathbb{E}[p_1(t) \mid q_1(0), p_1(0)].
\end{equation*}
The approximate equations of motion become:
\begin{align*}
    \frac{d}{dt} Q_1 &= P_1,\\
\frac{d}{dt} P_1 &= -Q_1\left(1 + \frac{1}{1 + Q_1^2} \right) - t \cdot \frac{2 Q_1^2 P_1}{(1 + Q_1^2)^2}. \end{align*}
{2 Q_1^2 P_1}{(1 + Q_1^2)^2}.
\end{align*}
The above equation can be solved numerically, and from it we obtain the Mori-Zwanzig approximation for our problem of interest.
\newpage
\nocite{*}
\printbibliography
\end{document}
    