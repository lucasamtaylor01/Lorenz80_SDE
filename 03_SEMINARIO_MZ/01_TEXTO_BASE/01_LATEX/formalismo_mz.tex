\documentclass[12pt]{article}

% Codificação e Idioma
\usepackage[utf8]{inputenc}
\usepackage[portuguese]{babel}

% Formatação de Página e Estilo de Texto
\usepackage[a4paper, margin=2cm]{geometry}
\usepackage{parskip}
\linespread{1.5}
\usepackage{csquotes}

% Links
\usepackage{xcolor}
\usepackage[hidelinks]{hyperref}

% Matemática
\usepackage{amsmath, amssymb, amsfonts}
\usepackage{cancel}
\usepackage{tikz}

% Bibliografia (ABNT)
\usepackage[backend=biber, style=abnt, language=portuguese, natbib=true]{biblatex}
\addbibresource{ref.bib}

% Outros Pacotes
\usepackage{graphicx}
\usepackage{ascii}
\usepackage{listings}
\usepackage{enumitem}
\usepackage{url}

\title{Introdução ao Formalismo Mori-Zwanzig}
\date{Abril de 2025}

\author{
    \textbf{Aluno:} Lucas Amaral Taylor\\
    \textbf{Orientador:} Prof. Dr. Breno Raphaldini Ferreira da Silva
}

\begin{document}
\maketitle

\section{Introdução}
O principal objetivo do artigo de \citet{Chekroun2021} é simplificar o modelo de Lorenz 80, preservando seu comportamento. Para isso, utilizaremos o método de Mori-Zwanzig, que é uma abordagem física-estatística aplicável em sistemas como o L80.

O método de Mori-Zwanzig, desenvolvido por Robert Walter Zwanzig e Hajime Mori na segunda metade do século XX, é utilizado em sistemas hamiltonianos. Esse método consiste em classificar as variáveis do sistema em duas categorias: ``resolvidas'' e ``não resolvidas''. As variáveis resolvidas são aquelas cujos comportamentos e valores são bem conhecidos, enquanto as não resolvidas são aquelas para as quais não se possui informações diretas. Para substituir essas variáveis não resolvidas, o método introduz termos estocásticos, denominados ruídos (\textit{noise}), além de um termo de amortecimento (\textit{damping}), também conhecido como termo de memória (\textit{memory term}). Essa abordagem permite que o comportamento do sistema de interesse seja preservado de maneira adequada, mesmo sem conhecer completamente as variáveis não resolvidas.

Dada a relevância deste método para o trabalho de \citet{Chekroun2021}, optamos por incluir uma introdução ao formalismo de Mori-Zwanzig, a fim de proporcionar uma melhor compreensão de sua aplicação no contexto do modelo de Lorenz 80 e permitir uma base teórica sólida para eventuais explorações.

\section{Motivação}
% Consideremos o seguinte exemplo de \citet{Chorin2013} como motivação: considere um sistema de duas partículas, em um espaço unidimencional com o hamiltoniado dado por:
% \begin{equation*}
%     H = \frac{1}{2}\left(q_1 ^2 + q_2^2 + q_1^2q_2^2 + p_1^2 + p_2^2\right)
% \end{equation*}
% onde:
% \begin{align*}
%     q_1, q_2 &\text{ são posições das partículas 1 e 2, respectivamente}\\
%     p_1, p_2 &\text{ são os momentos das partículas 1 e 2, respectivamente}
% \end{align*}

% O sistema é descrito por:
% \begin{align*}
%     \dot{q_1} &= p_1,\\
%     \dot{p_1} &= -q_1(1+q_2^2),\\
%     \dot{q_2} &= p_2,\\
%     \dot{p_2} &= -q_2(1+q_1^2).
% \end{align*}

\section{Prolegômenos}
\subsection{Escrevendo sistemas de EDO não lineares como sistemas de EDPs lineares}

Considere o sistema de equações diferenciais ordinárias (EDO) dado por:
\begin{equation}
    \frac{d}{dt} \phi(x,t) = R(\phi(x,t)), \quad \phi(x, 0) = x,
    \label{eq:prolegomenos-sistema-edo}
\end{equation}

onde $R$ é uma função não linear, $\phi$ é uma função dependente do tempo, e $R$, $\phi$ e $x$ podem assumir dimensões infinitas, sendo formados pelos vetores $R_i$, $\phi_i$ e $x_i$, respectivamente.

A partir disso, podemos definir o \textit{Operador de Liouville} associado à equação \eqref{eq:prolegomenos-sistema-edo} como:
\begin{equation}
    L = \sum_i R_i(x) \frac{\partial}{\partial x_i}
    \label{eq:prolegomenos-liouville-operator}
\end{equation}

Utilizando o \textit{Operador de Liouville}, podemos transformar o sistema de EDOs não lineares em um sistema de equações diferenciais parciais (EDPs) lineares da forma:
\begin{equation}
    u_t = Lu, \quad u(x,0) = g(x)
    \label{eq:prolegomenos-sistema-edp-linear}
\end{equation}

A solução desse sistema existe, é única, e é dada por $u(x,t) = g(\phi(x,t))$. Portanto, temos que a equação \eqref{eq:prolegomenos-sistema-edp-linear} é bem definida\footnote{Detalhes da demonstração podem ser encontrados em \citet[p.~181-182]{Chorin2013}}.

\subsection{Notação de semigrupo}
\subsubsection{Introdução à notação}
Tomemos $\Delta$ o operador de segunda derivada na variável espacial $x$: 
\begin{equation*}
    \Delta\psi = \frac{\partial^2 \psi}{\partial x^2}, \quad \text{ onde $\psi$ é uma função suave}
\end{equation*}

Assim como a solução da equação $v' - av = 0$, $v(0) = v_0$, $a = \text{constante}$, é dada por $e^{at}v_0$, escrevemos simbolicamente a solução da equação do calor $v_t - \frac{1}{2}\Delta v = 0$, com a condição inicial $v(x, 0) = \phi(x)$, como:
\begin{equation*}
    v(t) = e^{\frac{1}{2}t\Delta}\phi \quad \text{(essa é a notação de \textit{semigrupo})}.
\end{equation*}


\section{Mori-Zwanzig}
\subsection{Definição}
\subsection{Exemplos e propriedades}

\appendix
\section{Sistemas Hamiltonianos}

\section{Casos particulares apresentados}

\section{Aproximações}
\newpage
\nocite{*}
\printbibliography
\end{document}
