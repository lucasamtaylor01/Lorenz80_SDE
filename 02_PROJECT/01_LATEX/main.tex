\documentclass[12pt]{article}

\usepackage[utf8]{inputenc}
\usepackage[english]{babel}

\usepackage[a4paper, left=3cm, right=2cm, top=3cm, bottom=2cm]{geometry}
\usepackage{setspace}
\usepackage{parskip}

\usepackage{amsmath, amssymb}
\usepackage{graphicx}
\usepackage{enumitem}
\usepackage{booktabs}
\usepackage{lipsum}
\usepackage{csquotes}
\usepackage{ragged2e}
\usepackage{hyperref} 
\usepackage[style=apa, natbib=true]{biblatex}
\addbibresource{references.bib}

\setlength\bibitemsep{0pt}
\setlength\bibhang{1.25em}
\setlength\bibindent{0em}
\renewcommand*{\bibfont}{\fontsize{10pt}{12pt}\selectfont}


\begin{document}

\begin{titlepage}
	\centering
	{\Large\scshape MAP2419 -- Introduction to the Graduation Project \par}
	\vspace{0.3cm}
	{\large Institute of Mathematics and Statistics, University of São Paulo (IME-USP)\par}
		    
	\vspace{3cm}
		    
	{\LARGE\bfseries Project\par}
	\vspace{1cm}
	{\LARGE\bfseries A Stochastic Approach to the Lorenz 80 Model\par}
	\vspace{0.5em}
	{\large\bfseries Uma abordagem estocástica do Modelo de Lorenz 80\par}
		
	\vfill
		    
	\begin{minipage}[t]{0.45\textwidth}
		\raggedright
		\textbf{Supervisor:} \par
		Prof. Dr. Breno Raphaldini Ferreira da Silva \par
		\texttt{brenorfs@gmail.com} \par
		\textit{IME-USP} \par\medskip
	\end{minipage}
	\hfill
	\begin{minipage}[t]{0.45\textwidth}
		\raggedright
		\textbf{Student:} \par
		Lucas Amaral Taylor \par
		NUSP: 13865062 \par
		\texttt{lucasamtaylor@usp.br} \par
		\textit{IME-USP}
	\end{minipage}
	\vspace{2cm}
\end{titlepage}

\newpage

\begin{abstract}
	This project presents a study of the Lorenz 80 Model, originally proposed by Edward Lorenz (1980), from a stochastic approach inspired by \citet{Chekroun2021}. Essential theoretical foundations are covered, such as the Mori-Zwanzig formalism, general properties of stochastic differential equations and the mathematical and physical characteristics of the model itself. The computational development includes the implementation and numerical simulation of the model using the \textit{Julia} and \textit{Python} languages, with an emphasis on scientific libraries aimed at stochastic dynamic systems and data analysis. Finally, an exploratory analysis of different configurations of the noise term is carried out, with the aim of investigating alternative and complementary approaches to the original treatment.
\end{abstract}

\newpage

\section*{Introduction}

In 1980, Edward N. Lorenz published the article \textit{Attractor Sets and Quasi-Geostrophic Equilibrium} \citep{Lorenz1980}, in which he developed the model that became known as the \textit{Lorenz 80 Model} (L80), with the aim of studying the dynamics of forced and dissipative atmospheric systems.

Starting from shallow water equations with topography, Lorenz constructed a low-order model with nine ordinary differential equations, based on primitive equations. Subsequently, by eliminating the time derivative terms in the divergence equations, a quasi-geostrophic version with only three equations was obtained. The model presents two distinct scales of motion: fast oscillations, associated with gravitational waves, and slow oscillations, of a quasi-geostrophic nature. Over time, the fast components dissipate, and the dynamics concentrate on a reduced-dimension invariant manifold, where quasi-geostrophic equilibrium is a good approximation. In addition to the theoretical formulation, Lorenz performed computational simulations of the model, a feat of great importance for the time and for the development of computational mathematics.

In November 2021, the article \textit{Stochastic Rectification of Fast Oscillations on Slow Manifold Closures} \citep{Chekroun2021} was published, proposing a stochastic approach to slow-fast systems using methods from statistical physics. In this study, the authors used the \textit{L80 Model} as a case study, applying the Mori-Zwanzig (MZ) method.

The MZ method, which originated in statistical physics, separates the dynamics of a system into relevant and irrelevant parts using projection operators. A central feature of this method is that, when projecting the dynamics onto a subspace of relevant variables, the effects of the discarded variables do not disappear. The irrelevant variables are incorporated into the effective dynamics in the form of two additional terms: a Markovian term, which represents the influence of the memory of past states, and a noise term, which models the unresolved variability.

In January 2025, student Lucas Amaral Taylor took the course \textbf{MAP5007 - Waves in Geophysical Fluids}, offered in the summer program of the Institute of Mathematics and Statistics (IME-USP) and taught by Prof. Dr. Breno Raphaldini Ferreira da Silva. The course aimed to present basic concepts of geophysical fluid dynamics through a mathematical approach \citep{uspJanus}. At the end of the course, the student held a seminar\footnote{The seminar files are publicly available in the \textit{GitHub} repository \citep{TaylorL80}.} on the topic \textit{A brief study of the Lorenz 80 Model}, whose objective was to present the general aspects of the L80 model.

Finally, this work is an extension of that previous study. Now, instead of exploring the deterministic model of \citet{Lorenz1980}, the focus will be on the stochastic approach proposed by \citet{Chekroun2021}. Mathematical, statistical, and physical properties involved in the construction and treatment of the model will be analyzed, and computational simulations will be performed.


\section*{Objectives}
This work is based on three main objectives:
\begin{enumerate}
	\item \textbf{Understanding and manipulation of essential theoretical concepts}
	      \begin{enumerate}
	      	\item In-depth study of the MZ method: theory and applications, particularly in slow-fast dynamic systems;
	      	\item Analysis of the general properties of stochastic differential equations;
	      	\item Study of the particularities of the L80 Model, both in its deterministic and stochastic versions, considering its physical and mathematical implications.
	      \end{enumerate}
	      
	\item \textbf{Development of skills in computational tools.}
	      \begin{enumerate}
	      	\item Mastery of computational languages aimed at the simulation and analysis of mathematical models, especially Julia and Python, with a focus on the use of scientific libraries;
	      	\item Ability to implement, optimize, and interpret computational routines for numerical simulations.
	      \end{enumerate}
	      
	\item \textbf{Achievements in L80 model simulations and exploratory analysis.}
	      \begin{enumerate}
	      	\item Development of the L80 model in Julia;
	      	\item Exploratory noise analysis.
	      \end{enumerate}
\end{enumerate}


\section*{Methodology}
To understand and manipulate the mathematical objects fundamental to the development of the model, we began with an in-depth reading of the basic articles that introduce the main concepts and motivate the study, in particular the works of \citet{Chekroun2017} and \citet{Chekroun2021}. Next, to consolidate our understanding of the MZ method, we will analyze references that deal with the formulation and applications of this method, including the texts by \citet{Gouasmi2017}, \citet{Chorin2000}, \citet{Chorin2002}, and \citet{Chorin2013}.

In the development of computational tool skills, the Julia language will be used primarily, with a focus on the SciML library, aimed at simulations with stochastic differential equations, and Python, mainly with the numpy libraries.{SciML} \citep{SDEJulia}, geared toward simulations with stochastic differential equations, and Python, mainly with the libraries \texttt{numpy} and \texttt{pandas} for data analysis. For initial familiarization, simulations will be performed based on example 11.7 from \citep[p.~169]{Pavliotis2008}.

Finally, in the stage of \textbf{simulations of the L80 model and exploratory analysis}, computational simulations of the L80 model will be performed. Initially, we will reproduce the results presented in \citet{Chekroun2021}. Then, based on the knowledge and experience acquired throughout the project, we will ideally propose variations in noise modeling, different from those adopted in the original article. The intention is to assess whether alternative or complementary approaches to noise can improve the results obtained in the stochastic formulation.

\section*{Work plan}
\begin{center}
	\renewcommand{\arraystretch}{1.5}
	\begin{tabular}{p{3cm}p{10cm}}
		\toprule
		\textbf{Month} & \textbf{Activity}                                                                    \\
		\midrule
		April          & Definition of the theme, choice of advisor, and survey of the main references        \\
		May            & Introduction to the MZ method and the \textit{Julia} language.                       \\
		June           & First simulations in Julia using simplified models.                                  \\
		July           & Reading about stochastic differential equations and in-depth study of the L80 Model. \\ 
		August         & Initial implementation of the L80 Model.                                             \\
		September      & Exploratory analysis of noise term properties.                                       \\
		October        & Writing and completion of the monograph.                                             \\
		November       & Final review, translation, and preparation for presentation.                         \\
		\bottomrule
	\end{tabular}
\end{center}

\section*{Preliminary results}
The progress of the project can be followed via the link below:

\href{https://github.com/lucasamtaylor01/Lorenz80_SDE}{https://github.com/lucasamtaylor01/Lorenz80\_SDE}


\newpage
\nocite{*}
\printbibliography[title={References}, label={sec:bib}]

\end{document}