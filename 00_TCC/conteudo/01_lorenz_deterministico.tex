\chapter{O modelo de Lorenz 80 determinístico} \label{cap:ch01_lorenz_deterministico}

\section{Introdução} \label{sec:ch01_introducao}
Este capítulo tem como objetivo apresentar o modelo determinístico Lorenz 80 (L80). Para isso, iniciamos, na seção \ref{sec:ch01_geofisica}, com uma introdução aos conceitos básicos de geofísica, buscando familiarizar o leitor com os fundamentos desta área. Na sequência, na seção \ref{sec:ch01_apresentacao_do_modelo}, contextualizamos o modelo, abordando os trabalhos que o antecederam e as motivações que levaram à sua formulação.

A construção do modelo é detalhada na seção \ref{sec:ch01_construcao_do_modelo}, seguida pela apresentação de suas principais propriedades e características na seção \ref{sec:ch01_prop_do_modelo}. Por fim, na seção \ref{sec:ch01_simulacoes_deterministico}, são exibidas simulações computacionais realizadas com o modelo, acompanhadas da análise gráfica dos resultados.


\section{Breves considerações sobre geofísica} \label{sec:ch01_geofisica}

Nesta seção, reunimos um breve glossário com os principais conceitos de geofísica que servem de base para a compreensão do modelo de Lorenz 80 (L80). Como esse modelo se apoia no modelo de água rasa, também apresentaremos os fundamentos desse modelo.

\subsection{Glossário} \label{subsec:ch01_glossario}
Segue abaixo a definição de alguns conceitos de geofísica fundamentais para o modelo estudado:
\begin{itemize}
    \item \textbf{Quasi-geostrófico.}
    \item \textbf{Parâmetro de Coriolis.}
    \item \textbf{Equilíbrio hidrostático.}
\end{itemize}


\subsection{O modelo de água rasa} \label{subsec:ch01_agua_rasa}
O modelo de água rasa 
\section{Motivação e apresentação do modelo} \label{sec:ch01_apresentacao_do_modelo}

Edward Norton Lorenz (1917-2008) foi um importante matemático e meteorologista responsável pela publicação de vários artigos e desenvolvimento de modelos na área de previsão do tempo e outros fenômenos geofísicos. Dentre eles, o mais famoso foi o modelo Lorenz 63, conhecido popularmente pelo ``efeito borboleta'' que foi um marco de grande relevância para simulações matemáticas computacionais.

Em 1980, Lorenz publica o artigo intitulado ``\textit{Attractor Sets and Quasi-Geostrophic Equilibrium }'' \citep{Lorenz1980}. Nele, Lorenz apresenta a construção e a simulação de dois modelos distintos: o primeiro, é formado a partir das equações primitivas (PE) com nove EDOs (equações diferenciais ordinárias), derivado das equações de águas rasas com topografia e forçamento, enquanto o segundo é um modelo quasi-geostrófico (QG) com 3 EDOs, obtido ao descartar as variáveis associadas ao escoamento divergente $x$ e seus termos correspondentes. O modelo PE contém tanto ondas gravitacionais rápidas quanto oscilações quasi-geostróficas lentas, enquanto o modelo QG mantém apenas estas últimas, em um quadro simplificado para atmosfera de latitudes médias.

O artigo 

\section{Construção do modelo} \label{sec:ch01_construcao_do_modelo}
Como dito anteriormente, o modelo é construído a partir das equações de água-rasa com algumas particularidades descritas a seguir.

Consideremos um fluido homogêneo e incompressível, ou seja, com densidade constante em todo o volume e volume invariável mesmo sob variações de pressão. O escoamento é predominantemente horizontal, descrito por uma velocidade $V(t, \mathbf{r})$ independente da altura, onde $\mathbf{r}$ representa o vetor de posição inicial.

A componente vertical da velocidade é determinada pela continuidade de massa. A superfície livre do fluido está localizada na altura $H + z(t, \mathbf{r})$, onde $H$ representa a profundidade média e a base se apoia sobre uma topografia variável $h(\mathbf{r})$. Temos também que $h(\mathbf{r})$ e $z(t, \mathbf{r})$ possuem média zero.

O sistema está sujeito à rotação planetária, com um parâmetro de Coriolis constante $f$. 
Tanto o campo de velocidades $V$ quanto a elevação da superfície $z$ sofrem dissipação difusiva, 
associada a movimentos de pequena escala: o termo $\nu$ representa o coeficiente de difusão viscosa (dissipação de momento) e $\kappa$ representa o coeficiente de difusão térmica. O modelo também inclui um termo de forçamento externo $F(\mathbf{r})$ e, por fim, adota-se a hipótese de equilíbrio hidrostático.


A partir da descrição acima, podemos construir o seguinte diagrama:
\begin{figure}[H]
	\centering
	\begin{tikzpicture}[scale=1.5] 
				
		% Superfície livre
		\draw[above] (0,4) to[out=0,in=180] (1,4.2) 
		to[out=0,in=180] (2,3.8)
		to[out=0,in=180] (3,4.2)
		to[out=0,in=180] (4,3.8)
		to[out=0,in=180] (5,4.2)
		to[out=0,in=180] (6,3.8);
		\node[right] at (6,3.8) {Superfície livre};
				
		% Linha tracejada de referência para z
		\draw[dashed] (2,4.2) -- (1,4.2);
				
		% Desvio da superfície z
		\draw[thick] (2,3.8) -- (2,4.2);
		\draw[thick] (1.9,3.8) -- (2.1,3.8); % traço inferior
		\draw[thick] (1.9,4.2) -- (2.1,4.2); % traço superior
		\node[above] at (2,4.3) {$z(t,\mathbf{r})$};
				
		% Velocidade V(t,r)
		\node at (2,3) {$ V(t,\mathbf{r}) \rightarrow$};
				
		% Profundidade média H
		\draw[thick] (4,3.8) -- (4,1.8);
		\draw[thick] (3.9,3.8) -- (4.1,3.8); % traço superior
		\draw[thick] (3.9,1.8) -- (4.1,1.8); % traço inferior
		\node at (4.5,2.8) {$H$};
				
		% Topografia do fundo (curva inferior)
		\draw[dotted, thick] (0,2) to[out=10,in=170] (2,2.3) 
		to[out=350,in=170] (4,1.8) 
		to[out=350,in=170] (6,2);
		\node[right] at (6,1.9) {Topografia do fundo};
				
		% Variação da topografia h
		\draw[thick] (2,2.3) -- (2,1.8);
		\draw[thick] (1.9,2.3) -- (2.1,2.3); % traço superior
		\draw[thick] (1.9,1.8) -- (2.1,1.8); % traço inferior
		\node[below] at (2,1.6) {$h(\mathbf{r})$};
				
		% Linha tracejada de referência para h
		\draw[dashed] (2,1.8) -- (4,1.8);
				
	\end{tikzpicture}
	\caption{Diagrama do modelo de água-rasa adaptado}
	\label{fig:fluido-topografia}
\end{figure}

Além disso, o modelo de água-rasa adaptado é expresso por:
\begin{align}
	\frac{\partial V}{\partial t} & = - ( V \cdot \nabla)V - f \mathbf{k} \times V - g \nabla z + \nu \nabla^2 V \label{eq:agua-rasa-modificada-1}     \\
	\frac{\partial z}{\partial t} & = - (V \cdot \nabla)(z - h) - (H + z - h)\nabla \cdot  V + \kappa \nabla^2 z + F \label{eq:agua-rasa-modificada-2} 
\end{align}

% Onde:
% \begin{multicols}{2}
%     \begin{itemize}
%     	\item $t$: tempo
%     	\item $\mathbf{r}$: vetor de posição inicial;
%     	\item $H$: profundidade média do fluido;
%     	\item $h(\mathbf{r})$: variação da superfície topológica;
%     	\item $V(t,\mathbf{r})$: Velocidade horizontal;
%     	\item $z(t,\mathbf{r})$: altura da superfície;
%     	\item $f$: parâmetro de Coriolis;
%     	\item $g$: aceleração da gravidade;
%     	\item $F$: forças externas;
%     	\item $\kappa$: coeficiente de difusão viscosa;
%     	\item $\nu$: coeficiente de difusão térmica;
%     	\item $\mathbf{k}$: vetor da vertical.
%     \end{itemize}
% \end{multicols}

Em seguida, aplicamos a \textit{decomposição de Helmholtz} à equação \eqref{eq:agua-rasa-modificada-1}, escrevendo
\begin{equation*}
	V = \nabla \chi + \mathbf{k} \times \nabla \psi,
\end{equation*}
onde $\chi$ é o potencial de velocidade associado à parte divergente e $\psi$ a função corrente associada à parte rotacional. Dessa forma, $\nabla^2 \chi$ representa a divergência e $\nabla^2 \psi$ a vorticidade. Substituindo essa decomposição obtemos:
\begin{align}
	\frac{\partial \nabla^2 \chi}{\partial t} & = -\tfrac{1}{2}\nabla^2(\nabla \chi \cdot \nabla \chi) 
	- \nabla \chi \cdot \nabla(\nabla^2\psi) \times \mathbf{k} 
	+ \nabla^2(\nabla \chi \cdot \nabla \psi \times \mathbf{k}) \nonumber \\
    & \quad + \nabla \cdot (\nabla^2\psi\nabla\psi)          
	- \tfrac{1}{2}\nabla^2(\nabla \psi \cdot \nabla \psi) 
	+ \nu\nabla^4\chi + f\nabla^2\psi - g\nabla^2z, \label{eq:equacao-basica-1} \\
	\frac{\partial \nabla^2 \psi}{\partial t} & = -\nabla \cdot (\nabla^2\psi\nabla \chi)              
	- \nabla \psi \cdot \nabla(\nabla^2\psi) \times \mathbf{k} 
	- f\nabla^2\chi + \nu\nabla^4\psi. \label{eq:equacao-basica-2}
\end{align}

Analogamente, aplicando \eqref{eq:agua-rasa-modificada-2}, temos:
\begin{equation}
	\frac{\partial z}{\partial t} 
	= -\nabla \cdot \big[(z - h)\nabla \chi\big] 
	- \nabla \psi \cdot \nabla(z - h) \times \mathbf{k} 
	- H\nabla^2\chi + \kappa\nabla^2z + F. 
	\label{eq:equacao-basica-3}
\end{equation}

Nosso objetivo é reduzir as equações \eqref{eq:equacao-basica-1}–\eqref{eq:equacao-basica-3} a um modelo de baixa ordem. Para isso, introduzimos três vetores adimensionais $\alpha_1, \alpha_2, \alpha_3$ que satisfazem
\begin{equation*}
	\alpha_1 + \alpha_2 + \alpha_3 = 0,
\end{equation*}
e adotamos as permutações cíclicas
\begin{equation*}
	(i,j,k) = (1,2,3),\; (2,3,1),\; (3,1,2).
\end{equation*}
Definimos então:
\begin{equation*}
	a_i = \alpha_i \cdot \alpha_i, 
	\quad b_i = \alpha_j \cdot \alpha_k, 
	\quad c = (b_1b_2+b_2b_3+b_3b_1)^{1/2}.
\end{equation*}

Lorenz também apresenta uma forma alternativa, equivalente, mais conveniente para a implementação computacional:
\begin{equation*}
	b_i = \tfrac{1}{2}(a_i - a_j - a_k), 
	\quad c_i = c.
\end{equation*}

Escolhido um comprimento característico $L$, construímos três funções ortogonais:
\begin{equation*}
	\phi_i(\mathbf{r}) = \cos\!\left(\alpha_i \cdot \frac{\mathbf{r}}{L}\right),
\end{equation*}
para as quais valem, por exemplo:
\begin{align*}
	L^2\nabla^2\phi_i                              & = -a_i\phi_i,                         \\
	L^2\nabla\phi_i \cdot \nabla\phi_k             & = -\tfrac{1}{2}b_{ik}\phi_i + \cdots, \\
	L^2\nabla \cdot (\phi_j\nabla\phi_k)           & = \tfrac{1}{2}b_{jk}\phi_i + \cdots,  \\
	L^2\phi_j \cdot \nabla\phi_k \times \mathbf{k} & = -\tfrac{1}{2}c_{jk}\phi_i + \cdots, 
\end{align*}
onde os termos omitidos são múltiplos de cossenos. Com essas funções, expandimos as variáveis em série e introduzimos escalas adimensionais:
\begin{align*}
	t    & = f^{-1}\tau,               \\
	\chi & = 2L^2f^2 \sum_i x_i\phi_i,                                              \\
	\psi & = 2L^2f^2 \sum_i y_i\phi_i,                                              \\
	z    & = 2L^2f^2g^{-1} \sum_i z_i\phi_i,                                        \\
	h    & = 2L^2f^2g^{-1} \sum_i h_i\phi_i,                                        \\
	F    & = 2L^2f^2g^{-1} \sum_i F_i\phi_i.
\end{align*}

Substituindo as equações acima em \eqref{eq:equacao-basica-1}–\eqref{eq:equacao-basica-3}, e projetando sobre a base $\{\phi_i\}$, obtemos finalmente o modelo PE de baixa ordem, composto de nove equações diferenciais ordinárias:
\begin{align}
    a_i\frac{dx_i}{d\tau} & = a_ib_ix_ix_k - c(a_i - a_k)x_iy_k      
	+ c(a_i - a_j)y_ix_k -2c^2y_iy_k - \nu_0a_i^2x_i + a_iy_i - a_iz_i, \label{eq:modelo-pe-1}\\
	a_i\frac{dy_i}{d\tau} & = -a_ib_kx_iy_k - a_ib_iy_ix_k           
	+ c(a_k - a_i)y_iy_k - a_ix_i - \nu_0a_i^2y_i, \label{eq:modelo-pe-2}\\
	\frac{dz_i}{d\tau}    & = -b_kx_i(z_k - h_k) - b_i(z_i - h_i)x_k 
	+ c\,y_i(z_k - h_k) - c(z_i - h_i)y_k + g_0a_ix_i - \kappa_0a_iz_i + F_i. \label{eq:modelo-pe-3}
\end{align}

% onde,
% \begin{itemize}
% 	\item $x_i$ representam os modos divergentes do escoamento (associados às ondas de gravidade);
% 	\item $y_i$ representam os modos rotacionais (vorticidade), associados às oscilações quasi-geostróficas;
% 	\item $z_i$ são variáveis auxiliares acopladas ao sistema;
% \end{itemize}
Na construção do modelo QG, começamos desprezando todos os termos não lineares, assim como aqueles que envolvem as variáveis $x$, incluindo a derivada temporal, na equação \eqref{eq:modelo-pe-1}. Fazemos o mesmo com os termos não lineares ou topográficos que dependem de $x$ nas equações \eqref{eq:modelo-pe-2} e \eqref{eq:modelo-pe-3}. Por fim, eliminamos as variáveis $x$ e $z$, obtendo ao modelo QG apresentado a seguir:
\begin{equation}
	(a_i g_0 + 1)\,\frac{dy_i}{d\tau} 
	= g_0 c (a_k - a_j) y_j y_k 
	- a_i (a_i g_0 \nu_0 + \kappa_0) y_i 
	- c h_k y_j + c h_j y_k + F_i,
	\label{eq:modelo_lorenz_deterministico_qg}
\end{equation}

\section{Propriedades} \label{sec:ch01_prop_do_modelo}

\section{Simulações} \label{sec:ch01_simulacoes_deterministico}
\subsection{Escolha de parâmetros} \label{subsec:ch01_escolha_parametros}
\subsection{Gráficos} \label{subsec:ch01_graficos_simulacao}
