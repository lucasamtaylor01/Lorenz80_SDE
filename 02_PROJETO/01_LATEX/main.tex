\documentclass[12pt]{article}

% Codificação e idioma
\usepackage[utf8]{inputenc}
\usepackage[brazilian]{babel}
\linespread{1.5}


% Layout da página e espaçamento
\usepackage[a4paper, left=3cm, right=2cm, top=3cm, bottom=2cm]{geometry}
\usepackage{setspace}
\usepackage{parskip}

% Matemática
\usepackage{amsmath, amssymb}

% Gráficos e imagens
\usepackage{graphicx}

% Listas e tabelas
\usepackage{enumitem}
\usepackage{booktabs}

% Texto fictício
\usepackage{lipsum}
\usepackage{csquotes}
\usepackage{ragged2e}

% Citações e bibliografia
\usepackage[
    style=authoryear-comp,
    backend=biber,
    maxcitenames=2,
    maxbibnames=99,
    uniquelist=false,
    dashed=false,
    doi=true,
    url=true,
    giveninits=true,
    uniquename=init,
    sorting=nyt,
    language=portuguese,
    natbib=true
]{biblatex}
\addbibresource{referencias.bib}

\setlength\bibitemsep{0pt}               % Espaçamento entre itens (0pt = simples)
\setlength\bibhang{1.25em}               % Recuo da segunda linha
\setlength\bibindent{0em}                % Sem indentação na primeira linha
\renewcommand*{\bibfont}{\fontsize{10pt}{12pt}\selectfont}  % Tamanho da fonte 10pt com linha base de 12pt

\newenvironment{resumo}{
  \fontsize{10pt}{12pt}\selectfont  % tamanho 10 pt, altura de linha 12pt (simples)
  \setstretch{1}                    % espaçamento simples
  \par\noindent                     % começa parágrafo sem indentação extra
  \justifying                      % comando para justificar texto
}{
  \par
}

\begin{document}

\begin{titlepage}
	\centering
	{\Large\scshape MAP2419 -- Introdução ao Trabalho de Formatura \par}
	\vspace{0.3cm}
	{\large Instituto de Matemática e Estatística da Universidade de São Paulo (IME-USP)\par}
	    
	\vspace{3cm}
	    
	{\LARGE\bfseries Projeto\par}
	\vspace{1cm}
	{\LARGE\bfseries Uma abordagem estocástica do Modelo de Lorenz 80\par}
	\vspace{0.5em}
	{\large\bfseries A stochastic approach for the Lorenz 80 model\par}
	
	\vfill
	    
	\begin{minipage}[t]{0.45\textwidth}
		\raggedright
		\textbf{Orientador:} \par
		Prof. Dr. Breno Raphaldini Ferreira da Silva  \par
		\texttt{brenorfs@gmail.com} \par
		\textit{IME-USP} \par\medskip
	\end{minipage}
	\hfill
	\begin{minipage}[t]{0.45\textwidth}
		\raggedright
		\textbf{Aluno:} \par
		Lucas Amaral Taylor \par
		NUSP: 13865062 \par
		\texttt{lucasamtaylor@usp.br} \par
		\textit{IME-USP}
	\end{minipage}
	\vspace{2cm}
\end{titlepage}

\newpage
\section*{Resumo}
\begin{resumo}
    Este trabalho apresenta um estudo do Modelo Lorenz 80, originalmente proposto por Edward Lorenz (1980), a partir de uma abordagem estocástica inspirada em \citet{Chekroun2021}. São abordados fundamentos teóricos essenciais, como o formalismo de Mori-Zwanzig, propriedades gerais de equações diferenciais estocásticas e as características matemáticas e físicas do próprio modelo. O desenvolvimento computacional inclui a implementação e simulação numérica do modelo utilizando as linguagens \textit{Julia} e \textit{Python}, com ênfase em bibliotecas científicas voltadas para sistemas dinâmicos estocásticos e análise de dados. Por fim, realiza-se uma análise exploratória de diferentes configurações do termo de ruído, com o objetivo de investigar abordagens alternativas e complementares ao tratamento original.
\end{resumo}

\section*{Abstract}
\begin{resumo}
    This project presents a study of the Lorenz 80 Model, originally proposed by Edward Lorenz (1980), from a stochastic approach inspired by \citet{Chekroun2021}. Essential theoretical foundations are covered, such as the Mori-Zwanzig formalism, general properties of stochastic differential equations and the mathematical and physical characteristics of the model itself. The computational development includes the implementation and numerical simulation of the model using the \textit{Julia} and \textit{Python} languages, with an emphasis on scientific libraries aimed at stochastic dynamic systems and data analysis. Finally, an exploratory analysis of different configurations of the noise term is carried out, with the aim of investigating alternative and complementary approaches to the original treatment.
\end{resumo}

\newpage

\section*{Introdução}

Em 1980, Edward N. Lorenz publicou o artigo \textit{Attractor Sets and Quasi-Geostrophic Equilibrium} \citep{Lorenz1980}, no qual desenvolveu o modelo que ficou conhecido como \textit{Modelo Lorenz 80} (L80), com o objetivo de estudar a dinâmica de sistemas atmosféricos forçados e dissipativos.

Partindo das equações de águas rasas com topografia, Lorenz construiu um modelo de baixa ordem com nove equações diferenciais ordinárias, baseado em equações primitivas. Posteriormente, ao eliminar os termos de derivadas temporais nas equações de divergência, obteve-se uma versão quasi-geostrófica com apenas três equações. O modelo apresenta duas escalas distintas de movimento: oscilações rápidas, associadas a ondas gravitacionais, e oscilações lentas, de caráter quasi-geostrófico. Com o tempo, as componentes rápidas se dissipam, e a dinâmica concentra-se em uma variedade invariante de dimensão reduzida, onde o equilíbrio quasi-geostrófico é uma boa aproximação. Além da formulação teórica, Lorenz realizou simulações computacionais do modelo, um feito de grande importância para a época e para o desenvolvimento da matemática computacional.

Em novembro de 2021, foi publicado o artigo \textit{Stochastic Rectification of Fast Oscillations on Slow Manifold Closures} \citep{Chekroun2021}, que propôs uma abordagem estocástica para sistemas lentos-rápidos utilizando métodos da física estatística. Neste estudo, os autores utilizaram o \textit{Modelo L80} como estudo de caso, aplicando o método de Mori-Zwanzig (MZ).

O método MZ, originado na física estatística, separa a dinâmica de um sistema em partes relevantes e irrelevantes por meio de operadores de projeção. Uma característica central desse método é que, ao projetar a dinâmica sobre um subespaço de variáveis relevantes, os efeitos das variáveis descartadas não desaparecem. As variávies irrelevantes são incorporados à dinâmica efetiva na forma de dois termos adicionais: um termo markoviano, que representa a influência da memória dos estados passados, e um termo de ruído, que modela a variabilidade não resolvida.

Em janeiro de 2025, o aluno Lucas Amaral Taylor cursou a disciplina \textbf{MAP5007 - Ondas em Fluidos Geofísicos}, oferecida no programa de verão do Instituto de Matemática e Estatística (IME-USP) e ministrada pelo Prof. Dr. Breno Raphaldini Ferreira da Silva. A disciplina teve como objetivo apresentar conceitos básicos da dinâmica de fluidos geofísicos por meio de uma abordagem matemática \citep{uspJanus}. Ao final da disciplina, o aluno realizou um seminário\footnote{Os arquivos do seminário estão disponíveis publicamente no repositório do \textit{GitHub} \citep{TaylorL80}.} com o tema \textit{Um breve estudo do Modelo Lorenz 80}, cujo objetivo foi apresentar os aspectos gerais do modelo L80.

Por fim, o presente trabalho é uma extensão desse estudo anterior. Agora, em vez de explorar o modelo determinístico de \citet{Lorenz1980}, o foco será a abordagem estocástica proposta por \citet{Chekroun2021}. Serão analisadas propriedades matemáticas, estatísticas e físicas envolvidas na construção e tratamento do modelo, além de serem realizadas simulações computacionais.

    
\section*{Objetivos}
O presente trabalho, baseia-se em três principais objetivos:
\begin{enumerate}
	\item \textbf{Compreensão e manipulação dos conceitos teóricos essenciais}
	      \begin{enumerate}
	      	\item Estudo aprofundado do método MZ: teoria e aplicações, particularmente, em sistemas dinâmicos lentos-rápidos;
	      	\item Análise das propriedades gerais de equações diferenciais estocásticas;
	      	\item Estudo das particularidades do Modelo L80, tanto em seua versão determinística quanto na versão estocástica, considerando suas implicações físicas e matemáticas.
	      \end{enumerate}
	      
	\item \textbf{Desenvolvimento de habilidades em ferramentas computacionais.}
	      \begin{enumerate}
	      	\item Domínio de linguagens computacionais voltadas à simulação e análise de modelos matemáticos, especialmente \textit{Julia} e \textit{Python}, com foco na utilização de bibliotecas científicas;
	      	\item Capacidade de implementar, otimizar e interpretar rotinas computacionais para simulações numéricas.
	      \end{enumerate}
	      
	\item \textbf{Realizações de simulações do modelo L80 e análise exploratória.}
	      \begin{enumerate}
	      	\item Desenvolvimento do modelo L80 em \textit{Julia};
	      	\item Análise exploratória de ruídos.
	      \end{enumerate}
\end{enumerate}


\section*{Metodologia}
Para a \textbf{compreensão e manipulação dos objetos matemáticos fundamentais para o desenvolvimento do modelo}, iniciamos com a leitura aprofundada dos artigos-base que introduzem os principais conceitos e motivam o estudo, em especial os trabalhos de \citet{Chekroun2017} e \citet{Chekroun2021}. Em seguida, para consolidar o entendimento do método MZ, serão analisadas referências que tratam da formulação e aplicações deste método, incluindo os textos de \citet{Gouasmi2017}, \citet{Chorin2000}, \citet{Chorin2002} e \citet{Chorin2013}.

No \textbf{desenvolvimento de habilidades em ferramentas computacionais}, será utilizada, principalmente, a linguagem \textit{Julia} \citep{julialang}, com foco na biblioteca \texttt{SciML} \citep{SDEJulia}, voltada para simulações com equações diferenciais estocásticas e Python \citep{Python}, principalmente, com as bibliotecas \texttt{numpy} \citep{numpy} e \texttt{pandas} \citep{pandas} para análise de dados. Para a familiarização inicial, serão feitas simulações com base no exemplo 11.7 de \citep[p.~169]{Pavliotis2008}.

Por fim, na etapa de \textbf{realizações de simulações do modelo L80 e análise exploratória}, serão realizadas simulações computacionais do modelo L80. Inicialmente, reproduziremos os resultados apresentados em \citet{Chekroun2021}. Em seguida, com base no conhecimento e na experiência adquiridos ao longo do projeto, idealmente, proporemos variações na modelagem do ruído, distintas daquelas adotadas no artigo original. A intenção é avaliar se abordagens alternativas ou complementares em relação ao ruído podem aperfeiçoar os resultados obtidos na formulação estocásticas.

\section*{Plano de trabalho}
\begin{center}
	\renewcommand{\arraystretch}{1.5}
	\begin{tabular}{p{3cm}p{10cm}}
		\toprule
		\textbf{Mês} & \textbf{Atividade}\\
		\midrule
		Abril         & Definição do tema, escolha do orientador e levantamento das principais referências\\
		Maio          & Introdução ao método MZ e à linguagem \textit{Julia}.\\
		Junho         & Primeiras simulações em \textit{Julia} utilizando modelos simplificados.\\
		Julho         & Leitura sobre equações diferenciais estocásticas e aprofundamento no Modelo L80.\\
		Agosto        & Implementação inicial do Modelo L80.\\
		Setembro      & Análise exploratória das propriedades do termo de ruído.\\
		Outubro       & Redação e conclusão da monografia.\\
		Novembro      & Revisão final, tradução e preparação para a apresentação.\\
		\bottomrule
	\end{tabular}
\end{center}

\section*{Resultados preliminares}
O andamento do projeto pode ser acompanhado pelo link abaixo:

\url{https://github.com/lucasamtaylor01/Lorenz80_SDE}

\newpage
\nocite{*}
\printbibliography[title={Referências}, label={sec:bib}]

\end{document}